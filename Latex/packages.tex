%-------------------------------------FONT
\usepackage[T1]{fontenc}
\usepackage[utf8]{inputenc}
\usepackage{newlfont} % per frontespizio

%-------------------------------------LINGUE
\usepackage[english,
            italian]{babel}
\usepackage[autostyle,
            italian=guillemets]{csquotes}
% per virgolette corrette
% autostyle adatta lo stile delle citazioni alla lingua corrente del documento
% italian=guillemets racchiude automaticamente tra virgolette caporali i campi che prevedono le virgolette

%-------------------------------------BIBLIOGRAFIA
\usepackage[backend=biber,
            bibstyle=numeric,
            citestyle=numeric,
            style=numeric,
            sorting=none,
            backref,
            hyperref]{biblatex}
% Biber come motore bibliografico, è nuovo e da preferire a BibTex
% sorting=none: elenco delle citazioni nella bibliografia=nell'ordine in cui vengono citate nel testo
% backref: indica accanto a ciascun riferimento le pagine del documento in cui è citato
\addbibresource{biblio.bib}

%-------------------------------------BORDI
\usepackage{geometry} % permette la modifica della gabbia del documento
\geometry{
    paper=a4paper, % formato di pagina
    onecolumn,
    twoside,
    %top=5cm, % margine superiore
    bottom=4cm, % margine inferiore circa 5.5cm default
    %inner=2.5cm, % margine interno
    %outer=3cm, % margine esterno
    %heasheight= cm, % altezza della testatina
    %headsep=0.7cm, % altezza della separazione tra fondo della testatina e corpo del testo default circa 0.7cm
    heightrounded, % modifica di poco le dimensioni della gabbia per contenere un numero intero di righe
    hmargin=2.6cm, % dimensioni margini destro-sinistro
    %vmargin=2.5cm, % dimensioni margini superiore-inferiore
    %bindingoffset=0.5cm, % offset rilegatura
    %showframe, % uncomment to show how the type block is set on the page
}

%\usepackage{layaureo}
% Pacchetto layaureo per la formattazione italiana

%-------------------------------------IMMAGINI
\usepackage{graphicx}
% serve per includere immagini e grafici
\graphicspath{{res/fig/}}
% importa la cartella res/fig/ come cartella da cui caricare le immagini

\usepackage[inkscapelatex=false]{svg}
% permette di caricare immagini svg
\svgpath{{res/fig/}} % cartella .svg

%\usepackage{flafter}
% impedisce alle figure di apparire prima della loro definizione nel testo

%\usepackage{float}
% permette di forzare il posizionamento dell’oggetto nel punto in cui è situato con l’opzione H

%-------------------------------------TABELLE
\usepackage{tabularray} % tabelle belle
\usepackage{multirow} % tabelle con celle a mezza riga
\usepackage{booktabs} % tabelle con filetti belli

%-------------------------------------MATEMATICA
\usepackage{amsmath}
\usepackage{amsfonts}
\usepackage{amssymb}
\usepackage{xfrac} % nice inline one character fractions

\usepackage{physics}
% per scrivere derivate, ma con \qty crea incompatibilità con siunitx

\usepackage[output-decimal-marker={.}]{siunitx}
\sisetup{separate-uncertainty,
         per-mode = symbol} % incertezze col \pm
% siunitx: numeri con unità di misura
% output-decimal-marker={,}: le convenzioni tipografiche italiane prevedono la virgola e non il punto
\AtBeginDocument{\RenewCommandCopy\qty\SI}
% per usare qty di siunitx e non avere conflitti con physics
\DeclareSIUnit\clight{\text {c}}
% definizione dell'utente della velocità della luce c

%-------------------------------------CHIMICA
\usepackage{chemformula}

%-------------------------------------ALTRI
%\usepackage{setspace} % serve a fornire comandi di interlinea standard
%\onehalfspacing{} % imposta interlinea a 1,5 ed equivale a \linespread{1,5}

\usepackage{xcolor}

%\usepackage[a-1b]{pdfx} % conformità pdf generato

%\usepackage{lineno} % per numerare le linee di testo
%\linenumbers

\usepackage{comment}

\usepackage{microtype}

\usepackage{url}

%-------------------------------------LISTATI CODICE
\usepackage{listings}

\lstdefinelanguage{Python}{
  morekeywords={from, import, def, return},
  comment=[l]{\#},
  morestring=[b]",
  alsodigit={-},
  alsoletter={&},
}

\lstdefinestyle{myPython}{ %
  language=Python,                    
  % il linguaggio del codice
  basicstyle=\ttfamily\small,  
  % la dimensione dei font usati per il codice
  numbers=left,                       
  % dove mettere i numeri di linea
  numberstyle=\tiny\color{gray},      
  % lo stile usato per i numeri di linea
  stepnumber=1,                       
  % l'intervallo tra due numeri di linea. Se è 1, ogni linea sarà numerata
  numbersep=5pt,                      
  % quanto distano i numeri di linea dal codice
  backgroundcolor=\color{white},      
  % scegli il colore di sfondo. Devi aggiungere \usepackage{color}
  showspaces=false,                   
  % mostra gli spazi aggiungendo particolari underscore
  showstringspaces=false,             
  % sottolinea gli spazi all'interno delle stringhe
  showtabs=false,                     
  % mostra i tab all'interno delle stringhe aggiungendo particolari underscore
  %frame=single,                       
  % aggiunge un bordo attorno al codice
  rulecolor=\color{black},            
  % se non impostato, il colore del bordo potrebbe cambiare nelle interruzioni di riga
  tabsize=4,                          
  % imposta la dimensione del tab a 4 spazi
  captionpos=b,                       
  % imposta la posizione della didascalia in basso
  breaklines=true,                    
  % imposta l'interruzione automatica delle righe
  breakatwhitespace=false,            
  % imposta se le interruzioni automatiche dovrebbero avvenire solo agli spazi
  title=\lstname,                     
  % mostra il nome del file dei file inclusi con \lstinputlisting; prova anche caption al posto di title
  keywordstyle=\color{blue},          
  % stile delle parole chiave
  commentstyle=\color{green},         
  % stile dei commenti
  stringstyle=\color{red},            
  % stile delle stringhe letterali
}

%-------------------------------------Definizioni di comandi e ambienti
\newenvironment{abstract}
% definizione ambiente abstract ispirato ad article, non presente in book
  {%\cleardoublepage% se non si vuole sommario come un chapter
    \thispagestyle{empty}%
    \null \vfill\begin{center}%
      \chapter*{\abstractname} \end{center}}%
  {\vfill\null}

\usepackage{fancyhdr}
% per intestazioni e piè di pagina (fancy header)
\newcommand{\fncyfront}{%
    \fancyhead[RO]{{\footnotesize\rightmark}}
    \fancyfoot[RO]{\thepage}
    \fancyhead[LE]{\footnotesize{\leftmark}}
    \fancyfoot[LE]{\thepage}
    \fancyhead[RE,LO]{}
    \fancyfoot[C]{}
    \renewcommand{\headrulewidth}{0.3pt}}

\newcommand{\fncymain}{%
    \fancyhead[RO]{{\footnotesize\rightmark}}
    \fancyfoot[RO]{\thepage}
    \fancyhead[LE]{{\footnotesize\leftmark}}
    \fancyfoot[LE]{\thepage}
    \fancyfoot[C]{}
    \renewcommand{\headrulewidth}{0.3pt}}

%-------------------------------------RIFERIMENTI
\usepackage{hyperref} % DA CARICARE PER ULTIMO

% ANTO
% \usepackage[colorlinks=true,
%     linkcolor=black,
%     urlcolor=teal,
%     citecolor=black]{hyperref}

% \hypersetup{
%     colorlinks=true,
%     linkcolor=black,
%     filecolor=magenta,      
%     urlcolor=cyan,
%     pdftitle={Overleaf Example},
%     pdfpagemode=FullScreen,
% }